\documentclass{article}
\usepackage[a4paper, margin=3cm]{geometry} % Marges instellen op 2 cm rondom
\usepackage{fancyhdr} % Voor paginanummering en headers/footers
\usepackage{graphicx} % Voor het invoegen van afbeeldingen
\setlength{\parindent}{0pt} % Geen inspringing voor alinea's
\setlength{\parskip}{1em}   % Extra ruimte tussen alinea's
\usepackage{placeins}  % Voeg dit toe aan je preamble

% Pagina-instellingen
\pagestyle{fancy}
\fancyhf{}
\fancyfoot[C]{\thepage} % Paginanummering onderaan in het midden

\title{Portabolt Research } % Hier vul je de titel van je document in
\author{Tom Bergisch , Maikel van Es} % Hier vul je de naam van de auteur in
\date{10 september 2023} % Vaste datum toevoegen

\begin{document}

\maketitle % Voeg de titel, auteur en datum toe

\tableofcontents % Voeg een automatische inhoudsopgave toe
\newpage % Start de tekst op een nieuwe pagina na de inhoudsopgave

\section{Inleiding}
In deze sectie wordt een inleiding gegeven tot het project. Hier kun je uitleg geven over de achtergrond, doelstellingen, en algemene context van het project.

Het project beoogt een gedetailleerde beschrijving en ontwikkeling van verschillende werkpakketten die bijdragen aan de succesvolle uitvoering van het MIT AIP Grant Project.

% Invoegen van werkpakketbestanden
\section{Werkpakket 1:  Scoping en modelling}
In dit werkpakket zal eerst verder onderzocht worden hoe het verdienmodel er exact uitziet. In de keten zit een afnemer, Portabolt met de hardware en AIP met de connectie naar de beurs. Er zal grondig onderzoek nodig zijn om te achterhalen hoe elke schakel in de keten voldoende waarde krijgt. Er zal ook van tevoren moeten worden nagedacht over hoe elke laag met elkaar communiceert en hierover moeten afspraken gemaakt worden. Er zal in kaart gebracht worden hoe een prototype er precies uitziet en hoe we kunnen bewijzen dat alle stappen die gezet moeten worden haalbaar zijn. 

Deliverables: 
\begin{enumerate}
	\item Uitgewerkte verdienmodellen per doelgroep en partner
	\item Functioneel ontwerp contractuele randvoorwaarden van dienstverlening 
	\item Interfaces documentatie (tussen partners) 
	\item Functioneel ontwerp prototype 
\end{enumerate}

Uitgavenplanning begroting: 

Uren begroot: 740


%\subsection{Uitgewerkte verdienmodellen per doelgroep en partner}

\subsection{Uitgewerkte verdienmodellen per doelgroep en partner}

\subsubsection{Identificeer doelgroepen en partners}
Begin met het identificeren van alle belangrijke doelgroepen en partners die betrokken zijn bij het project. Dit kan onder meer de eindgebruikers (doelgroepen), leveranciers, tussenpersonen en andere relevante partijen omvatten. In dit specifieke geval zijn de belangrijkste partijen mogelijk de afnemers, Portabolt en AIP.

\subsubsection{Analyseer toegevoegde waarde}
Voor elke geïdentificeerde partij, analyseer en begrijp de specifieke waarde die zij toevoegen aan het product of de dienst. Wat maakt hun bijdrage uniek en waardevol? Hoe dragen ze bij aan het uiteindelijke succes van het project?

\subsubsection{Verdienmodel ontwerpen}
Portabolt Power is een verhuurbedrijf van modulaire mobiele batterijsystemen. In de verhuur draait alles om de bezettingsgraad. Er zijn voor Portabolt Power 3 verschillende businesscases.

-	De inzet van assets die niet in de verhuur staan 
-	De inzet van assets bij de klant
-	De volledige inzet van assets, stop van de verhuur

\paragraph{De inzet van assets die niet in de verhuur staan }
De bezettingsgraad van Portabolt Power ligt momenteel tussen de 50 en 60 procent. Voor deze analyse gaan we uit van een standaard asset 15 kVA met 40 kWh batterij. De verhuur opbrengsten van deze asset zijn 450 euro per week. 
Uitwerken verdienmodel in spreadsheet


\paragraph{De inzet van assets bij de klant}
Zodra ze bij de klant draaien kunnen ze ook ingezet worden om een stukje netcongestie te doen. Uitdaging daarbij is wel dat wanneer een systeem voor peakshaving ergens wordt neergezet de netaansluiting niet toereikend is, dus alleen voor specifieke toepassingen waarbij de peakload hoog is kan dit nuttig zijn. 

\paragraph{De volledige inzet van assets, stop van de verhuur}
Dit is naar verwachting de meest rendabele businesscase. Er is een omslagpunt vanaf wanneer de continue omzet van assets op de onbalans markt lucratiever wordt dan de losse verhuur van batterijen. Dit omslagpunt houden wij nauwlettend in de gaten. 


\subsubsection{Rekening houden met belangen}
Zorg ervoor dat het ontwikkelde verdienmodel recht doet aan de belangen en investeringen van elke partij. Het moet een win-winsituatie creëren waarbij alle betrokken partijen profiteren van de samenwerking.

Bijvoorbeeld: Het verdienmodel moet eerlijk zijn voor alle partijen en een evenwicht vinden tussen winstgevendheid en waardecreatie voor de klant.

\subsubsection{Documentatie en communicatie}
Documenteer het uitgewerkte verdienmodel duidelijk en communiceer het aan alle betrokken partijen. Zorg ervoor dat er een gemeenschappelijk begrip is van hoe elke partij zal profiteren en wat hun verantwoordelijkheden zijn.

Communiceer eventuele wijzigingen of updates in het verdienmodel tijdig en transparant om een soepele samenwerking te garanderen.

\subsection{Functioneel ontwerp contractuele randvoorwaarden van dienstverlening}

\subsubsection{Definieer contractuele verplichtingen}
Begin met het identificeren van alle relevante contractuele verplichtingen die moeten worden vastgelegd tussen de verschillende partners. Dit omvat zaken als serviceniveaus, prijsstructuren, aansprakelijkheid, vertrouwelijkheid en andere voorwaarden die van invloed zijn op de dienstverlening.

\subsubsection{Bepaal rechten en verantwoordelijkheden}
Specificeer duidelijk de rechten en verantwoordelijkheden van elke partij binnen de overeenkomst. Dit omvat wat elke partij moet leveren, wat ze kunnen verwachten en welke acties moeten worden ondernomen in verschillende scenario's.

\subsubsection{Juridische compliance en risicobeheer}
Zorg ervoor dat de contractuele randvoorwaarden voldoen aan alle relevante juridische vereisten en standaarden. Dit omvat het identificeren en beheren van eventuele juridische risico's die kunnen ontstaan uit de dienstverlening. 

Uitwerken van de iec normen die relevant zijn:

\subsubsection{Prijsstructuren en betalingsvoorwaarden}
Definieer duidelijk de prijsstructuren voor de diensten die worden geleverd, inclusief eventuele variabele kosten, abonnementskosten of andere tarieven. Specificeer ook de betalingsvoorwaarden, facturatiecycli en eventuele boetes of vergoedingen voor te late betalingen.

\subsubsection{Service Level Agreements (SLA's)}
Ontwikkel gedetailleerde Service Level Agreements (SLA's) die de kwaliteit, beschikbaarheid, prestaties en ondersteuningsniveaus van de dienstverlening definiëren. Deze SLA's moeten meetbaar en haalbaar zijn en moeten de verwachtingen van alle betrokken partijen weerspiegelen.

\subsubsection{Vertrouwelijkheid en gegevensbescherming}
Beheer vertrouwelijkheidsclausules en bepalingen met betrekking tot gegevensbescherming en privacy. Zorg ervoor dat gevoelige informatie adequaat wordt beschermd en dat er passende maatregelen zijn genomen om de privacy van gebruikers te waarborgen.

\subsubsection{Onderhandeling en goedkeuring}
Faciliteer onderhandelingen tussen de betrokken partijen om tot overeenstemming te komen over de contractuele randvoorwaarden. Zorg ervoor dat alle partijen zich gehoord voelen en dat eventuele geschillen of zorgen worden aangepakt voordat de overeenkomst definitief wordt goedgekeurd.

\subsubsection{Documentatie en ondertekening}
Documenteer de contractuele randvoorwaarden nauwkeurig en zorg ervoor dat alle betrokken partijen de overeenkomst begrijpen en akkoord gaan met de voorwaarden voordat ze worden ondertekend. Zorg ervoor dat er kopieën worden bewaard voor toekomstige referentie en naleving.

\subsection{Interfaces documentatie (tussen partners)}

\subsection{Identificeer de vereiste interfaces}
Identificeer alle interfaces die nodig zijn voor de communicatie tussen de partners. In dit geval is het essentieel om de API-specificaties te definiëren die Portabolt Power levert voor de aansturing van het batterijsysteem.

\subsubsection{Documenteer de API-specificaties}
Documenteer gedetailleerd de specificaties van de API die Portabolt Power levert. Dit omvat de beschikbare endpoints, ondersteunde HTTP-methoden (zoals GET, POST, PUT, DELETE), vereiste parameters, geaccepteerde invoerformaten (bijv. JSON), en de verwachte uitvoerformaten en responscodes.

\subsubsection{Authenticatie en autorisatie}
Specificeer de authenticatiemethoden en autorisatievereisten die nodig zijn om toegang te krijgen tot de API van Portabolt Power. Dit kan onder meer het gebruik van API-tokens, OAuth 2.0-authenticatie of andere mechanismen omvatten.

\subsubsection{Gegevensuitwisseling en protocollen}
Beschrijf hoe gegevens worden uitgewisseld tussen de partners via de API. Dit kan onder meer het gebruik van bepaalde gegevensindelingen, zoals JSON of XML, omvatten, evenals de communicatieprotocollen, zoals HTTPS, die worden gebruikt voor de beveiligde overdracht van gegevens.

\subsubsection{Foutafhandeling en logging}
Definieer hoe fouten worden afgehandeld en gelogd binnen de API. Dit omvat het vastleggen van foutmeldingen, het definiëren van standaardfoutcodes en het bieden van richtlijnen voor foutafhandeling aan de partners die de API gebruiken.

\subsubsection{Beveiliging en gegevensbescherming}
Adres beveiligingsmaatregelen zoals gegevensversleuteling, beveiligde overdracht van gegevens en toegangscontrolemechanismen om ervoor te zorgen dat de API van Portabolt Power voldoet aan de hoogste normen voor gegevensbescherming en privacy.

\subsubsection{Testen en validatie}
Ontwikkel testcases en validatiemethoden om ervoor te zorgen dat de API van Portabolt Power correct functioneert en voldoet aan de verwachtingen van de partners. Dit omvat het testen van verschillende scenario's en randgevallen om de robuustheid en betrouwbaarheid van de API te garanderen.

\subsubsection{Documentatie en ondersteuning}
Lever uitgebreide documentatie voor de API van Portabolt Power, inclusief handleidingen, voorbeelden en API-referenties, om partners te helpen bij het effectief integreren en gebruikmaken van de API. Bied ook technische ondersteuning en begeleiding aan partners indien nodig.
 % Inlezen van werkpakket 1
\section{Werkpakket 2: Titel van Werkpakket 2}
2. Prototyping V0.1 (4 maanden) 

Het doel van het prototype is om zo snel mogelijk alle afzonderlijke stappen te testen en bewijzen dat het werkt. Het zal in een virtuele omgeving draaien waarbij makkelijk geïtereerd kan worden. Portabolt zal een interface ontwikkelen waarmee extern de batterijen aangestuurd kunnen worden. Er zal een communicatielaag worden gebouwd waarmee AIP de batterijen kan aansturen. Er wordt een testprotocol ontwikkeld waarmee elke afzonderlijke stap getest en gevalideerd kan worden. 
Deliverables: 
1. Portabolt sturingsinterface 
2. Koppeling tussen All in power, Portabolt, Edge, Envitron 
3. Werkend prototype in gesimuleerde omgeving 

Uren begroot: 1720
 % Inlezen van werkpakket 2
\section{Werkpakket 3: Titel van Werkpakket 3}
Hier komt de beschrijving van werkpakket 3.

zijn er nog andere zaken die ik wil testen?  % Inlezen van werkpakket 3
\section{Werkpakket 4: Titel van Werkpakket 4}
Hier komt de beschrijving van werkpakket 4. % Inlezen van werkpakket 4
\section{Werkpakket 5: Titel van Werkpakket 5}
Hier komt de beschrijving van werkpakket 5. % Inlezen van werkpakket 5
\section{Werkpakket 6: Titel van Werkpakket 6}
Hier komt de beschrijving van werkpakket 6. % Inlezen van werkpakket 6
\section{Werkpakket 7: Titel van Werkpakket 7}
Hier komt de beschrijving van werkpakket 7. % Inlezen van werkpakket 7

\end{document}
